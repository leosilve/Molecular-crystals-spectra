\documentclass[pt12]{article}

\usepackage{graphicx}
%\usepackage{epsfig}
\usepackage{amsmath}

%input "C:\Documents and Settings\User\Documenti\My TeX Files\Bicocca\MC.bib"


\textwidth 16 cm \textheight 24 cm \oddsidemargin 0 cm \topmargin
-2cm
\newcommand{\beq}{\begin{equation}}
\newcommand{\eeq}{\end{equation}}
\newcommand{\bea}{\begin{eqnarray}}
\newcommand{\eea}{\end{eqnarray}}
\newcommand{\bfk}{\mathbf{k}}
\newcommand{\bfq}{\mathbf{q}}
\newcommand{\bfr}{\mathbf{r}}
\newcommand{\bfn}{\mathbf{n}}
\newcommand{\bfm}{\mathbf{m}}
\newcommand{\bfK}{\mathbf{K}}
\newcommand{\bfR}{\mathbf{R}}
\newcommand{\bfd}{\mathbf{d}}
\newcommand{\bfu}{\mathbf{u}}
\newcommand{\bfl}{\mathbf{l}}
\newcommand{\bfp}{\mathbf{p}}
\newcommand{\bfj}{\mathbf{j}}
\newcommand{\tmu}{\tilde{\mu}}
\newcommand{\tnu}{\tilde{\nu}}
\newcommand{\cm}{cm$^{-1}$ }
\newcommand{\unu}{\underline{\nu}}
\newcommand{\umu}{\underline{\mu}}


\begin{document}

%\include{MC.bib}

\title{Optical absorption and emission spectra of molecular crystals}
\author{Leonardo Silvestri}
\maketitle

%\begin{abstract}

%\end{abstract}

\section{Hamiltonian}

We assume an infinite 3D crystal in which each molecule has one
vibrational and one electronic degree of freedom. Molecules are
labeled according to the unit cell they belong to (indicated by
$\bfn$) and to their position inside the cell (indicated by
$\alpha$). Vibrationally each molecule $\bfn,\alpha$ has one
effective configuration coordinate $Q_{\bfn,\alpha}$, and the
vibrational potential is $V^{gr}_{\bfn,\alpha}=\hbar\Omega
Q_{\bfn,\alpha}^2$ in the electronic ground state and
$V^{ex}_{\bfn,\alpha}=\hbar\Omega\left(Q_{\bfn,\alpha}-\lambda\right)^2$
in the excited state. We include the interaction between all the
molecules of the infinite crystal; the Hamiltonian reads

\begin{equation}\label{H}
H^{FE}=H^{FE}_{elec}+H^{ph}+H^{FE-ph},
\end{equation}

with

\begin{equation}\label{H_FE}
H^{FE}_{elec}=\hbar\omega\sum_{\bfn,\alpha}a^\dagger_{\bfn,\alpha}a_{\bfn,\alpha}
+\sum_{                         \begin{array}{c}
                                  \bfn,\alpha,\bfm,\beta \\
                                  \{\bfn,\alpha\} \neq \{\bfm,\beta\} \\
                                \end{array}
}  J_{\alpha\beta}(\bfr_{\bfn}-\bfr_{\bfm})
a^\dagger_{\bfn,\alpha}a_{\bfm,\beta},
\end{equation}

\begin{equation}\label{H_ph}
H^{ph}=\hbar\Omega\sum_{\bfn,\alpha}b^\dagger_{\bfn,\alpha}b_{\bfn,\alpha},
\end{equation}

\begin{equation}\label{H_FE-ph}
H^{FE-ph}=\hbar\Omega\sum_{\bfn,\alpha}a^\dagger_{\bfn,\alpha}a_{\bfn,\alpha}
\left[-\lambda\left(b^\dagger_{\bfn,\alpha}+b_{\bfn,\alpha}\right)+\lambda^2\right].
\end{equation}

\section{Multi particle basis functions}

The multiparticle basis states are denoted by

\begin{eqnarray}\label{MP_states}
| \bfn,\alpha ; \underline{\nu}\rangle &\equiv& \\
\nonumber &\equiv& | \bfn,\alpha \rangle | \cdots
\nu_{\{-1,0,0\};1}\cdots\nu_{\{-1,0,0\};\sigma}
\nu_{\textbf{0};1}\cdots \tilde{\nu}_{\textbf{0};\alpha} \cdots
\nu_{\textbf{0};\sigma}
\nu_{\{1,0,0\};1}\cdots\nu_{\{1,0,0\};\sigma}\cdots \rangle \\
\nonumber &\equiv& a^\dagger_{\bfn,\alpha}|0^{el}\rangle \times
\frac{1}{\sqrt{\nu_{\textbf{0};\alpha}!}}\left(
\tilde{b}^\dagger_{\bfn,\alpha}\right)^{\nu_{\textbf{0};\alpha}}|\tilde{0}_{\bfn,\alpha}\rangle
\times \prod_{(\delta\bfn,\beta)\neq (\textbf{0};\alpha)}
\frac{1}{\sqrt{\nu_{\delta\bfn,\beta}!}}\left(
b^\dagger_{\bfn+\delta\bfn,\beta}\right)^{\nu_{\delta\bfn;\beta}}|0_{\bfn+\delta\bfn,\beta}\rangle,
\end{eqnarray}

where $\bfn\equiv\{n_a,n_b,n_c\}$ denotes the crystal vector
$\bfr_\bfn=n_a \textbf{a}+n_b \textbf{b}+n_c \textbf{c}$ (the
notation $\textbf{0}=\{0,0,0\}$ was also used);
$\alpha=1,\ldots,\sigma$ labels the translationally non-equivalent
molecules in the unit cell placed at
$\bfr_{\bfn,\alpha}=\bfr_\bfn+\bfr_\alpha$, where $\bfr_\alpha$
denotes the position inside the cell. Operators
$b^\dagger_{\bfn,\alpha}$ create a phonon excitation localized on a
$\alpha$ molecule in its ground state, while
$\tilde{b}^\dagger_{\bfn,\alpha}=b^\dagger_{\bfn,\alpha}-\lambda$
create a phonon excitation localized on a $\alpha$ molecule in its
excited state.

With the restriction that phonon clouds be localized around the
purely electronic excitation we Fourier transform the basis states
and we need a single wave vector that describes the delocalization
of the electronic excitation surrounded by the phonon cloud:

\begin{equation}\label{MP_ststes_FT}
| \bfk,\alpha ; \underline{\nu}\rangle=\frac{1}{\sqrt{N}}\sum_\bfn
\exp(i\bfk\bfr_\bfn)|\bfn,\alpha;\underline{\nu}\rangle.
\end{equation}


\subsection{Matrix elements in real space}

First we compute matrix elements in real space, i.e.

\begin{equation}\label{Matrix_elements_def_nm}
\langle \bfn,\alpha ; \underline{\nu} | H^{FE} | \bfm,\beta ;
\underline{\mu} \rangle.
\end{equation}

We separate the Hamiltonian into two terms and obtain respectively

\begin{equation}\label{Matrix_elements_ph_nm}
\langle \bfn,\alpha ; \underline{\nu} | H^{ph}+H^{FE-ph} |
\bfm,\beta ; \underline{\mu} \rangle = \hbar \Omega
\delta_{\{\bfn,\alpha\},\{\bfm,\beta\}} \delta_{\underline{\nu},
\underline{\mu}} \sum_{\delta\bfn,\gamma} \nu_{\delta\bfn,\gamma}
\end{equation}

and

\begin{eqnarray}\label{Matrix_elements_elec_nm}
\langle \bfn,\alpha ; \underline{\nu} | H^{FE}_{elec} | \bfm,\beta ;
\underline{\mu} \rangle &=& \hbar \omega
\delta_{\{\bfn,\alpha\},\{\bfm,\beta\}} \delta_{\underline{\nu},
\underline{\mu}} + \\
 + J_{\alpha\beta}(\delta\bfr) S\left(
                                  \begin{array}{c}
                                    \nu_{\textbf{0},\alpha} \\
                                    \mu_{\delta\bfr,\alpha} \\
                                  \end{array}
                                \right)
 S\left(
                                \begin{array}{c}
                                    \mu_{\textbf{0},\beta} \\
                                    \nu_{-\delta\bfr,\beta} \\
                                  \end{array}
                                \right)
& & \prod_{\{\bfr,\gamma\} \neq \{\textbf{0},\alpha\} \neq
\{-\delta\bfr,\beta\}} \langle
\nu_{\bfr,\gamma}|\mu_{\bfr+\delta\bfr,\gamma} \rangle,
\end{eqnarray}
where
\begin{eqnarray}\label{delta_n_def}
\delta\bfr\equiv \bfr_\bfn-\bfr_\bfm, \\
J_{\alpha\alpha}(\textbf{0})=0.
\end{eqnarray}
The symbol $\delta_{\underline{\nu}, \underline{\mu}}$ is different
from 0 only if the two phonon clouds are identical and localized on the same
molecule. Moreover we have

\begin{eqnarray}\label{S_def}
S\left(
                                \begin{array}{c}
                                    \mu \\
                                    \nu \\
                                  \end{array}
                                \right) &\equiv&
\left\langle \frac{1}{\sqrt{\mu !}}(b^\dagger)^\mu 0 \right|
\left.\frac{1}{\sqrt{\nu !}}(\tilde{b}^\dagger)^\nu 0 \right\rangle =\\
&=&
\frac{\exp(-\lambda^2/2)}{\sqrt{\mu!\nu!}}\sum_{i=0}^{\textrm{min}(\mu,\nu)}
\frac{(-1)^{\nu-i}\lambda^{\mu+\nu-2i}\mu!\nu!}{i!(\mu-i)!(\nu-i)!}
\end{eqnarray}
and
\begin{equation}\label{Ground_Vibrational_overlap}
\langle \nu_{\bfr,\gamma}|\mu_{\bfr+\delta\bfr,\gamma}
\rangle=\delta_{\nu,\mu}
\end{equation}



\subsection{Matrix elements in \textbf{k} space}

We want to compute

\begin{equation}\label{Matrix_elements_def_FT}
\langle \bfk,\alpha ; \underline{\nu} | H^{FE} | \bfk',\beta ;
\underline{\mu} \rangle
\end{equation}

Using the results  of the previous section and equation
(\ref{MP_ststes_FT}) we find

\begin{equation}\label{Matrix_elements_ph_FT}
\langle \bfk,\alpha ; \underline{\nu} | H^{ph}+H^{FE-ph} |
\bfk',\beta ; \underline{\mu} \rangle = \hbar \Omega
\delta_{\{\bfk,\alpha\},\{\bfk',\beta\}} \delta_{\underline{\nu},
\underline{\mu}} \sum_{\delta\bfn,\gamma} \nu_{\delta\bfn,\gamma}
\end{equation}

and

\begin{eqnarray}\label{Matrix_elements_elec_nm}
\langle \bfk,\alpha ; \underline{\nu} | H^{FE}_{elec} | \bfk',\beta ;
\underline{\mu} \rangle &=& \hbar \omega
\delta_{\{\bfk,\alpha\},\{\bfk',\beta\}} \delta_{\underline{\nu},
\underline{\mu}} + \\
\nonumber
 + \delta_{\bfk,\bfk'} \sum_{\delta\bfr}
J_{\alpha\beta}(\delta\bfr) \exp(i\bfk\delta\bfr)
 S\left(
                                  \begin{array}{c}
                                    \nu_{\textbf{0},\alpha} \\
                                    \mu_{\delta\bfr,\alpha} \\
                                  \end{array}
                                \right)
 S\left(
                                \begin{array}{c}
                                    \mu_{\textbf{0},\beta} \\
                                    \nu_{-\delta\bfr,\beta} \\
                                  \end{array}
                                \right)
& & \prod_{\{\bfr,\gamma\} \neq \{\textbf{0},\alpha\} \neq
\{-\delta\bfr,\beta\}} \langle
\nu_{\bfr,\gamma}|\mu_{\bfr+\delta\bfr,\gamma} \rangle,
\end{eqnarray}
where
\begin{equation}\label{delta_n_def}
J_{\alpha\alpha}(\textbf{0})=0.
\end{equation}

\section{Multi-phonon basis set}

\subsection{Real space}

We start with a basis set made by one-, two-, etc. particle states in real space. We include two intramolecular phonon modes: one low energy mode ($lp$) described by the multi-phonon basis set , and a second high energy mode ($hp$) included within a vibronic approximation. In our one-particle (\textbf{zero phonon}) states the electronic excitation and $\tmu$ high energy vibrations are localized on molecule $\alpha$ in the crystal cell $\bfn=(n_1,n_2,n_3)$ at position $\textbf{r}_{n,\alpha}$; they are
indicated as

\begin{equation}\label{0P_states}
\left|\textbf{n},\alpha, \tmu \right>=\left|\textbf{n},\alpha
\right>_e \otimes \left| \textbf{n},\alpha, \tmu \right>_{hp} \equiv a^\dagger_{\bfn,\alpha,\tmu} \left|0\right>.
\end{equation}

In our two-particle (\textbf{one phonon}) states the electronic excitation and the high energy vibrations are localized on molecule $\alpha$ in the crystal cell $\bfn$ at position $\textbf{r}_{n,\alpha}$, while a single low energy vibration is localized on a different molecule $\beta$ in the crystal cell $\bfm$ at position $\textbf{r}_{m,\beta}$; they are indicated as

\begin{equation}\label{1P_states}
\left|\textbf{n},\alpha, \tmu ; \textbf{m},\beta
\right>=\left|\textbf{n},\alpha \right>_e \otimes \left|
\textbf{n},\alpha, \tmu \right>_{hp} \otimes \left| \textbf{m},\beta
\right>_{lp}.
\end{equation}

In our three-particle (\textbf{two phonons}) states the electronic excitation and the high energy vibrations are localized on molecule $\alpha$ in the crystal cell $\bfn$ at position $\textbf{r}_{n,\alpha}$, while a single low energy vibration is localized on a different molecule $\beta_1$ in the crystal cell $\bfm_1$ at position $\textbf{r}_{m_1,\beta_1}$ and another single low energy vibration is localized on a different molecule $\beta_2$
in the crystal cell $\bfm_2$ at position $\textbf{r}_{m_2,\beta_2}$; they are indicated as

\begin{equation}\label{2P_states}
\left|\textbf{n},\alpha, \tmu ; \textbf{m}_1,\beta_1;
\textbf{m}_2,\beta_2 \right>=\left|\textbf{n},\alpha \right>_e
\otimes \left| \textbf{n},\alpha, \tmu \right>_{hp} \otimes \left|
\textbf{m}_1,\beta_1 \right>_{lp} \otimes \left|
\textbf{m}_2,\beta_2 \right>_{lp}.
\end{equation}

In the same way we can define \textbf{p-phonon} states. The
Hamiltonian in real space reads

\begin{equation}\label{H}
H^{\rm two-modes}=H^{FE}_{elec}+H^{ph}+H^{FE-ph},
\end{equation}

with

\begin{equation}
H^{FE}_{elec}=\left( \hbar\omega+\tmu\hbar\Omega_{hp}
\right)\sum_{\bfn,\alpha,\tmu}a^\dagger_{\bfn,\alpha,\tmu}a_{\bfn,\alpha,\tmu}
+\sum_{
\begin{array}{c}
                                  \bfn,\alpha,\bfm,\beta,\tmu \\
                                  \{\bfn,\alpha\} \neq \{\bfm,\beta\} \\
                                \end{array}
}  S^2\left(
                                  \begin{array}{c}
                                    \tmu \\
                                    0 \\
                                  \end{array}
                                \right)
J_{\alpha\beta}(\bfr_{\bfn}-\bfr_{\bfm})
a^\dagger_{\bfn,\alpha,\tmu}a_{\bfm,\beta,\tmu},
\end{equation}

\begin{equation}
H^{ph}=\hbar\Omega\sum_{\bfn,\alpha}b^\dagger_{\bfn,\alpha}b_{\bfn,\alpha},
\end{equation}

\begin{equation}\label{H_FE-ph}
H^{FE-ph}=\hbar\Omega\sum_{\bfn,\alpha}a^\dagger_{\bfn,\alpha}a_{\bfn,\alpha}
\left[-\lambda\left(b^\dagger_{\bfn,\alpha}+b_{\bfn,\alpha}\right)+\lambda^2\right].
\end{equation}


where we used the same notation as in section 1 for the low energy
mode and we added the high energy phonon term.

\subsection{Wave vector space}

\subsubsection{States}
In strong EP coupling regime is useful to turn to Fourier space and use BO basis functions made of products of Bloch combinations of both electronic and phononic parts:

\begin{equation}\label{0P_k_states}
\left|\textbf{k},\alpha, \tmu \right>=\frac{1}{\sqrt{N}}\sum_\bfn
\exp\left( i\bfk\bfr_{n,\alpha} \right)\left|\textbf{n},\alpha, \tmu
\right> \equiv a^\dagger_{\bfk,\alpha,\tmu} \left|0\right> ;
\end{equation}

\begin{equation}\label{1P_k_states}
\left|\textbf{k},\alpha, \tmu ; \textbf{q},\beta \right>=\frac{1}{N}
\sum_{\bfn,\bfm}{} \exp\left( i\bfk\bfr_{n,\alpha} +
i\bfq\bfr_{m,\beta} \right)\left|\textbf{n},\alpha, \tmu\right> b^\dagger_{\textbf{m},\beta}
\left|0\right>_{lp} = a^\dagger_{\bfk,\alpha,\tmu} b^\dagger_{\bfq,\beta}\left|0\right>,
\end{equation}

where

\begin{equation}\label{1P_k_states}
b^\dagger_{\bfq,\beta}\equiv \frac{1}{\sqrt{N}} \sum_{\bfm}{} \exp\left( i\bfq\bfr_{m,\beta} \right)b^\dagger_{\textbf{m},\beta} ;
\end{equation}

\begin{eqnarray}\label{2P_k_states}
\nonumber
\left|\textbf{k},\alpha, \tmu ; \textbf{q}_1,\beta_1 ;
\textbf{q}_2,\beta_2\right>&=&\frac{1}{N\sqrt{N}}
\sum_{\bfn,\bfm_1,\bfm_2}{} \exp\left( i\bfk\bfr_{n,\alpha} +
i\bfq_1\bfr_{m_1,\beta_1} + i\bfq_2\bfr_{m_2,\beta_2}
\right)\left|\textbf{n},\alpha, \tmu\right>
b^\dagger_{\textbf{m}_2,\beta_2}b^\dagger_{\textbf{m}_1,\beta_1}
\left|0\right>_{lp} \\
&=& a^\dagger_{\bfk,\alpha,\tmu} b^\dagger_{\bfq_2,\beta_2}b^\dagger_{\bfq_1,\beta_1}\left|0\right> ,
\end{eqnarray}
with the appropriate normalization factors: $\sqrt{2}^{-1}$ if
$\bfq_1=\bfq_2$ and $\beta_1=\beta_2$, 1 else;

\begin{eqnarray}\label{4P_k_states}
&&\left|\textbf{k},\alpha, \tmu ; \textbf{q}_1,\beta_1 ;
\textbf{q}_2,\beta_2; \textbf{q}_3,\beta_3\right>= \\ \nonumber
&&=\frac{1}{N^2} \sum_{\bfn,\bfm_1,\bfm_2,\bfm_3}{} \exp\left(
i\bfk\bfr_{n,\alpha} + i\bfq_1\bfr_{m_1,\beta_1} +
i\bfq_2\bfr_{m_2,\beta_2}+ i\bfq_3\bfr_{m_3,\beta_3}
\right)\left|\textbf{n},\alpha, \tmu\right>
b^\dagger_{\textbf{m}_3,\beta_3}b^\dagger_{\textbf{m}_2,\beta_2}b^\dagger_{\textbf{m}_1,\beta_1}
\left|0\right>_{lp}\\ \nonumber
&& = a^\dagger_{\bfk,\alpha,\tmu} b^\dagger_{\bfq_3,\beta_3} b^\dagger_{\bfq_2,\beta_2}b^\dagger_{\bfq_1,\beta_1}\left|0\right> ,
\end{eqnarray}
with the appropriate normalization factors: $\sqrt{6}^{-1}$ if all
three phonons are the same, $\sqrt{2}^{-1}$ if there are two equal
phonons, 1 else.

In the case of lattice with N cells and $\sigma$ molecules per cell
we have:

\begin{eqnarray}
\textrm{number of 0 phonon states:  } && \sigma ,\\
\textrm{number of 1 phonon states:  } && N \sigma^2 ,\\
\textrm{number of 2 phonon states:  } && \sigma \left[ N\cdot CR^2_\sigma +\sigma^2 C_N^2 \right] ,\\
\textrm{number of 3 phonon states:  } && \sigma \left[ N\cdot
CR_\sigma^3 + \sigma\cdot D_N^2\cdot CR_\sigma^2 + \sigma^3\cdot
C_N^3 \right] ,
\end{eqnarray}

where
\begin{eqnarray}
C_n^k&=&\left(
        \begin{array}{c}
          n \\
          k \\
        \end{array}
      \right) = \frac{n!}{k!(n-k)!} \\
CR_n^k &=& C_{n+k-1}^k=\frac{(n+k-1)!}{k!(n-1)!} \\
D_n^k &=& \frac{n!}{(n-k)!}.
\end{eqnarray}


\subsubsection{Hamiltonian}
The Hamiltonian can be written in Fourier space and it reads

\begin{eqnarray}\label{H_FT}
H&=&\left(\hbar\omega+\tmu\hbar\Omega_{hp}\right)\sum_{\bfk,\alpha,\tmu}
a^\dagger_{\bfk,\alpha,\tmu}a_{\bfk,\alpha,\tmu}
+\sum_{\bfk,\alpha,\beta,\tmu,\tmu'} S\left(
                                \begin{array}{c}
                                    \tmu \\
                                    0 \\
                                  \end{array}
                                \right)
S\left(
                                \begin{array}{c}
                                    \tmu' \\
                                    0 \\
                                  \end{array}
                                \right)
\tilde{J}_{\alpha\beta}(\bfk)
a^\dagger_{\bfk,\alpha,\tmu}a_{\bfk,\beta,\tmu'}\\
\nonumber &&+\hbar\Omega\sum_{\bfq,\alpha}
b^\dagger_{\bfq,\alpha}b_{\bfq,\alpha}+\lambda^2\hbar\Omega
-\frac{\lambda\hbar\Omega}{\sqrt{N}}\sum_{\bfk,\bfq,\alpha,\tmu}
\left[
b^\dagger_{\bfq,\alpha}a^\dagger_{\bfk,\alpha,\tmu}a_{\bfk+\bfq,\alpha,\tmu}
+
b_{\bfq,\alpha}a^\dagger_{\bfk,\alpha,\tmu}a_{\bfk-\bfq,\alpha,\tmu}
\right].
\end{eqnarray}

Hamiltonian matrix elements between
\textbf{0 phonon} states are:

\begin{eqnarray}\label{1P_H_1P}
\left< \bfk',\alpha',\tmu'\right| H \left|\bfk,\alpha,\tmu\right>
&=& \delta (\bfk-\bfk') \delta_{\alpha\alpha'} \delta_{\tmu\tmu'}
\left[\hbar \omega+\tmu \hbar \Omega_{hp}+\lambda^2 \hbar \Omega
\right] \\ \nonumber && + \delta (\bfk-\bfk') S\left(
                                \begin{array}{c}
                                    \tmu \\
                                    0 \\
                                  \end{array}
                                \right)
S\left(
                                \begin{array}{c}
                                    \tmu' \\
                                    0 \\
                                  \end{array}
                                \right)
\tilde{J}_{\alpha\alpha'}(\bfk)
\end{eqnarray}

where
\begin{equation}\label{Jk}
\tilde{J}_{\alpha\alpha'}(\bfk)=\sum_{\bfn}{}'
\exp\left[i\bfk(\bfr_{n,\alpha}-\bfr_{0,\alpha'})\right]
J_{\alpha\alpha'}(\bfr_{n,\alpha}-\bfr_{0,\alpha'}),
\end{equation}
and $\sum_{\bfn}{}'$ means that $\{n,\alpha\} \neq \{0,\alpha'\}$.

\textbf{Coupling between 0- and 1- phonon} states is given by:

\begin{equation}\label{1P_H_2P}
\left< \bfk',\alpha',\tmu'\right| H \left|\bfk,\alpha,\tmu; \bfq,
\beta\right> =
-\frac{\lambda\hbar\Omega}{\sqrt{N}}\delta_{\tmu\tmu'}\delta_{\alpha\alpha'}\delta_{\alpha\beta}
\delta(\bfk+\bfq-\bfk'),
\end{equation}

Hamiltonian matrix elements between \textbf{1-phonon states} are:

\begin{eqnarray}\label{2P_H_2P}
\nonumber \left< \bfk',\alpha',\tmu'; \bfq', \beta' \right| H
\left|\bfk,\alpha,\tmu; \bfq, \beta \right> &=& \delta (\bfk-\bfk')
\delta (\bfq-\bfq') \delta_{\beta\beta'} \left\{
\delta_{\alpha\alpha'} \delta_{\tmu\tmu'} \left[\hbar \omega + \tmu
\hbar \Omega_{hp} + \hbar \Omega \left( 1+\lambda^2\right) \right]
\right. \\
&& \left. +
S\left(
                                \begin{array}{c}
                                    \tmu \\
                                    0 \\
                                  \end{array}
                                \right)
S\left(
                                \begin{array}{c}
                                    \tmu' \\
                                    0 \\
                                  \end{array}
                                \right)
\tilde{J}_{\alpha\alpha'}(\bfk)\right\},
\end{eqnarray}


\textbf{Coupling between 1- and 2- phonon} states is given by:


\begin{eqnarray}\label{3P_H_2P}
\nonumber && \left< \bfk',\alpha',\tmu'; \bfq',\beta'\right| H
\left|\bfk,\alpha,\tmu; \bfq_1,\beta_1 ; \bfq_2,\beta_2\right> =
-\frac{\lambda\hbar\Omega}{\sqrt{N}}\delta_{\alpha\alpha'}
\delta_{\tmu\tmu'} \left[
\delta(\bfk-\bfk'+\bfq_2)\delta(\bfq_1-\bfq')\delta_{\alpha\beta_2}\delta_{\beta_1\beta'}
\right.  \\ && \left. +
\delta(\bfk-\bfk'+\bfq_1)\delta(\bfq_2-\bfq')\delta_{\alpha\beta_1}\delta_{\beta_2\beta'}
\right]\left[1+\delta_{\beta_1\beta_2}\delta(\bfq_1-\bfq_2)\left(\frac{1}{\sqrt{2}}-1\right)\right];
\end{eqnarray}

Hamiltonian matrix elements between \textbf{2-phonon states} are:

\begin{eqnarray}\label{3P_H_3P}
&& \left< \bfk',\alpha',\tmu'; \bfq_1',\beta_1' ; \bfq_2',\beta_2'
\right| H \left|\bfk,\alpha,\tmu;
\bfq_1,\beta_1 ; \bfq_2,\beta_2 \right> = \\
\nonumber && \delta (\bfk-\bfk') \left[ \delta (\bfq_1-\bfq_1')
\delta(\bfq_2-\bfq_2') \delta_{\beta_1\beta_1'}
\delta_{\beta_2\beta_2'} + \delta (\bfq_1-\bfq_2')
\delta(\bfq_2-\bfq_1') \delta_{\beta_1\beta_2'}
\delta_{\beta_2\beta_1'} - \right. \\
\nonumber && \left. \delta (\bfq_1-\bfq_1') \delta(\bfq_2-\bfq_2')
\delta_{\beta_1\beta_1'}
\delta_{\beta_2\beta_2'}\delta(\bfq_1-\bfq_2)
\delta_{\beta_1\beta_2}  \right]\\ \nonumber && \times
\left[\delta_{\alpha\alpha'} \delta_{\tmu\tmu'} \left[ \left( 2 +
\lambda^2 \right) \hbar \Omega + \tmu \hbar \Omega_{hp}+\hbar \omega
\right] + S\left(
                                \begin{array}{c}
                                    \tmu \\
                                    0 \\
                                  \end{array}
                                \right)
S\left(
                                \begin{array}{c}
                                    \tmu' \\
                                    0 \\
                                  \end{array}
                                \right)
\tilde{J}_{\alpha\alpha'}(\bfk)\right].
\end{eqnarray}

\textbf{Coupling between 2- and 3- phonon} states is given by:

\begin{eqnarray}\label{4P_H_3P}
&& \left< \bfk',\alpha',\tmu'; \bfq_1',\beta_1'; \bfq_2',\beta_2'
\right| H \left|\bfk,\alpha,\tmu; \bfq_1,\beta_1 ;
\bfq_2,\beta_2; \bfq_3,\beta_3 \right> = \\
\nonumber &&
-\frac{\lambda\hbar\Omega}{\sqrt{N}}\delta_{\alpha\alpha'}
\delta_{\tmu\tmu'} \left[ \right. \\
\nonumber && +
\delta(\bfk+\bfq_1-\bfk')\delta(\bfq_2-\bfq_1')\delta(\bfq_3-\bfq_2')
\delta_{\alpha\beta_1}\delta_{\beta_2\beta_1'}\delta_{\beta_3\beta_2'}
\\ \nonumber && +
\delta(\bfk+\bfq_1-\bfk')\delta(\bfq_3-\bfq_1')\delta(\bfq_2-\bfq_2')
\delta_{\alpha\beta_1}\delta_{\beta_3\beta_1'}\delta_{\beta_2\beta_2'}
\\ \nonumber && +
\delta(\bfk+\bfq_2-\bfk')\delta(\bfq_1-\bfq_1')\delta(\bfq_3-\bfq_2')
\delta_{\alpha\beta_2}\delta_{\beta_1\beta_1'}\delta_{\beta_3\beta_2'}
\\ \nonumber && +
\delta(\bfk+\bfq_2-\bfk')\delta(\bfq_3-\bfq_1')\delta(\bfq_1-\bfq_2')
\delta_{\alpha\beta_2}\delta_{\beta_3\beta_1'}\delta_{\beta_1\beta_2'}
\\ \nonumber && +
\delta(\bfk+\bfq_3-\bfk')\delta(\bfq_1-\bfq_1')\delta(\bfq_2-\bfq_2')
\delta_{\alpha\beta_3}\delta_{\beta_1\beta_1'}\delta_{\beta_2\beta_2'}
\\ \nonumber && +
\delta(\bfk+\bfq_3-\bfk')\delta(\bfq_2-\bfq_1')\delta(\bfq_1-\bfq_2')
\delta_{\alpha\beta_3}\delta_{\beta_2\beta_1'}\delta_{\beta_1\beta_2'}
\left.\right]\times
\\ \nonumber && \times
\left[1+\left(\frac{1}{\sqrt{2}}-1\right)\delta(\bfq_1'-\bfq_2')
\delta_{\beta_1'\beta_2'}\right] \times
\\ \nonumber && \times
\left[1+\left(\frac{1}{\sqrt{2}}-1\right)\delta(\bfq_1-\bfq_2)
\delta_{\beta_1\beta_2}+\left(\frac{1}{\sqrt{2}}-1\right)\delta(\bfq_1-\bfq_3)
\delta_{\beta_1\beta_3}+
\left(\frac{1}{\sqrt{2}}-1\right)\delta(\bfq_2-\bfq_3)
\delta_{\beta_2\beta_3}\right.
\\ \nonumber && +
\left(2-\frac{3}{\sqrt{2}}+\frac{1}{\sqrt{6}}\right)\delta(\bfq_1-\bfq_2)
\delta_{\beta_1\beta_2}\delta(\bfq_2-\bfq_3) \delta_{\beta_2\beta_3}
\left.\right]
\end{eqnarray}

Hamiltonian matrix elements between ordered (also the molecule part)
\textbf{3-phonon states} are:

\begin{eqnarray}\label{4P_H_4P}
&& \left< \bfk',\alpha',\tmu'; \bfq_1',\beta_1' ; \bfq_2',\beta_2';
\bfq_3',\beta_3' \right| H \left|\bfk,\alpha,\tmu;
\bfq_1,\beta_1 ; \bfq_2,\beta_2; \bfq_3,\beta_3 \right> = \\
\nonumber && \delta (\bfk-\bfk') \delta (\bfq_1-\bfq_1')
\delta(\bfq_2-\bfq_2') \delta(\bfq_3-\bfq_3')
\delta_{\beta_1\beta_1'} \delta_{\beta_2\beta_2'}
\delta_{\beta_3\beta_3'} \\ \nonumber && \times
\left[\delta_{\alpha\alpha'} \delta_{\tmu\tmu'} \left[ \left( 3 +
\lambda^2 \right) \hbar \Omega + \tmu \hbar \Omega_{hp}+\hbar \omega
\right] + S\left(
                                \begin{array}{c}
                                    \tmu \\
                                    0 \\
                                  \end{array}
                                \right)
S\left(
                                \begin{array}{c}
                                    \tmu' \\
                                    0 \\
                                  \end{array}
                                \right)
\tilde{J}_{\alpha\alpha'}(\bfk)\right].
\end{eqnarray}

We can also treat the $\bfq=0$ phonon exactly and write the
Hamiltonian as

\begin{eqnarray}\label{H_FT_q0}
H&=&\left(\hbar\omega+\tmu\hbar\Omega_{hp}\right)\sum_{\bfk,\alpha,\tmu}
a^\dagger_{\bfk,\alpha,\tmu}a_{\bfk,\alpha,\tmu}
+\sum_{\bfk,\alpha,\beta,\tmu,\tmu'} S\left(
                                \begin{array}{c}
                                    \tmu \\
                                    0 \\
                                  \end{array}
                                \right)
S\left(
                                \begin{array}{c}
                                    \tmu' \\
                                    0 \\
                                  \end{array}
                                \right)
\tilde{J}_{\alpha\beta}(\bfk)
a^\dagger_{\bfk,\alpha,\tmu}a_{\bfk,\beta,\tmu'}\\
\nonumber &&+\hbar\Omega\sum_{\alpha}\left[
\tilde{\tilde{b}}^\dagger_{0,\alpha}\tilde{\tilde{b}}_{0,\alpha}+\sum_{\bfq\neq
0}
b^\dagger_{\bfq,\alpha}b_{\bfq,\alpha}\right]+\left(1-\frac{1}{N}\right)\lambda^2\hbar\Omega
\\
\nonumber && -\frac{\lambda\hbar\Omega}{\sqrt{N}}\sum_{\bfk,\bfq\neq
0,\alpha,\tmu} \left[
b^\dagger_{\bfq,\alpha}a^\dagger_{\bfk,\alpha,\tmu}a_{\bfk+\bfq,\alpha,\tmu}
+
b_{\bfq,\alpha}a^\dagger_{\bfk,\alpha,\tmu}a_{\bfk-\bfq,\alpha,\tmu}
\right],
\end{eqnarray}

where

\begin{equation}\label{b0}
\tilde{\tilde{b}}_{0,\alpha} \equiv b_{0,\alpha}-
\frac{\lambda}{\sqrt{N}}.
\end{equation}


\subsection{Analysis of the Strong I regime for a planar 4T aggregate}

Suppose that we have a planar aggregate with N lattice cells, each
with 2 molecules, i.e. a total of 2N molecules. The two molecules
are changed one into another by the screw symmetry and we assume
that also their dipoles are related by the same symmetry operation.
With the dipole signs we adopted we find that the lower band is
symmetric under screw symmetry and it is b polarized, while the
upper band is anti-symmetric and ac polarized. In order to obtain
some analytical result we also suppose that the lower band is
completely flat, i.e. it has no dispersion in k, and it has energy 0
for each $\bfk_i$ with $i=1,...,N$; we also assume that the upper
band is much higher in energy, i.e. the Davydov splitting $DS$ is
much larger than $\hbar\Omega$. Here $i=1,...,N$ labels the N wave
vectors in the first Brillouin zone. We now consider 0, 1, 2 and 3
phonon states and write the Hamiltonian for all the states with
total wave vector equal to 0 built on the $N$ degenerate excitons
belonging to the lower band. We find that for each polarization
there is single 1 phonon state, a single 2 phonons state and, in the
limit of large N, a single 3 phonons state which couple to the
corresponding allowed 0 phonon state.

For b polarization the relevant matrix can be written (in units of
$\hbar\Omega$)

\begin{equation}\label{Matrix_b}
\left(
  \begin{array}{cccc}
    0 & -\lambda\frac{1}{\sqrt{2}} & 0 & 0\\
    -\lambda\frac{1}{\sqrt{2}} & 1 & -\lambda & 0\\
    0 & -\lambda & 2 & -\lambda \sqrt{\frac{3}{2}\left(1-\frac{1}{8N^2}\right)} \\
    0 & 0 & -\lambda \sqrt{\frac{3}{2}\left(1-\frac{1}{8N^2}\right)} & 3 \\
  \end{array}
\right);
\end{equation}

for ac polarization the relevant matrix can be written (in units of
$\hbar\Omega$)

\begin{equation}\label{Matrix_ac}
\left(
  \begin{array}{cccc}
    DS & -\lambda\frac{1}{\sqrt{2}} & 0 & 0 \\
    -\lambda\frac{1}{\sqrt{2}} & 1 & -\lambda\frac{1}{\sqrt{2}} & 0 \\
    0 & -\lambda\frac{1}{\sqrt{2}} & 2 & -\lambda \sqrt{1-\frac{3}{8N^2}} \\
    0 & 0 & -\lambda \sqrt{1-\frac{3}{8N^2}} & 3 \\
  \end{array}
\right),
\end{equation}

where $DS>>1$ is the purely excitonic Davydov splitting in units of
$\hbar\Omega$. From the above matrices is easy to compute the
position of the first three peaks in each polarization as a function
of the parameter $\lambda$. Moreover we see that within our
approximations the peak positions depend very little on the
aggregate size. In the limit of $DS\rightarrow \infty$ and
$N\rightarrow \infty$ the splitting between the first two replicas
changes from 0 to 0.49$\hbar\Omega$ as $\lambda$ changes from 0 to
2. For $\lambda=1.32$, we have $\beta$=173 cm$^{-1}$ and
$\delta$=198 cm$^{-1}$.

\subsubsection{Analytical details}

The above matrices have been found using the following procedure.
Consider a given polarization (for instance the b polarization), we
have a single 0 phonon state which is

\begin{equation}
\Psi^b_0=\frac{1}{\sqrt{2}}\left(
\left|\bfk'=0,\alpha'=0\right>+\left|\bfk=0,\alpha=1 \right>
\right).
\end{equation}

For states with a phonon part we can build a basis of states which
is diagonal in each subspace with a given number of phonons. If we
write the generic state as

\begin{equation}
\left|\bfk_i,\alpha; P \right>,
\end{equation}

where P denotes the phonon part of the state, the diagonal states
can be built as

\begin{equation}\label{diag_state}
\psi^b(i,P)=\frac{ \left|\bfk_i,\alpha; P
\right>+\left|\bfk_i,\alpha^*; P \right>+\left|\bfk_i^*,\alpha^*;
P^* \right>+\left|\bfk_i^*,\alpha; P^* \right>
}{|\left|\bfk_i,\alpha; P \right>+\left|\bfk_i,\alpha^*; P
\right>+\left|\bfk_i^*,\alpha^*; P^* \right>+\left|\bfk_i^*,\alpha;
P^* \right>  |} ,
\end{equation}

where the asterisks denotes here the screw symmetry operation which
acts as follows: $0^*=1$, $1^*=0$,
$\bfk^*=(k_x,k_y,k_z)^*=(-k_x,k_y,-k_z)$. We note that the first two
states appearing in the above expression interact through the
$\tilde{J}(\bfk)$ term of the Hamiltonian, while the third and the
fourth term are obtained from the first and the second respectively
two by applying the screw symmetry.


In the 1 phonon subspace we can build only N states $\psi^b_1(i)$
because the phonon part is completely determined by the choice of
$i$. It is easy to see that all the states $\psi^b_1(i)$ have the
same matrix element with $\Psi^b_0$, i.e. $\langle \psi^b_1(i) |
\Psi^b_0 \rangle = I_{01}$. The single 1 phonon state that couples
with $\Psi^b_0$ is therefore

\begin{equation}
\Psi^b_1=\frac{1}{\sqrt{N}} \sum_{i=1}^N \psi^b_1(i).
\end{equation}

This can be proved by constructing an orthonormal basis set in the 1
phonon subspace, such that for any other state $\Phi^b_1=\sum_i c_i
\psi^b_1(i)$, we have

\begin{equation}
\langle \Phi^b_1 | \Psi^b_1 \rangle = \sum_i \frac{1}{N} c_i =
\sum_i c_i = 0;
\end{equation}

from the above relation it follows that

\begin{equation}
\langle \Phi^b_1 | \Psi^b_0 \rangle = \sum_i c_i I_{01}= \sum_i c_i
= 0.
\end{equation}

We now turn to the \textbf{2 phonons subspace}. This subspace is
spanned by N(N+1) states $\psi^b_2(i, P)\equiv \psi^b_2(n)$, where
we label them with a single index $n=1,...,N(N+1)$ for clarity. We
can construct the single 2 phonons state that couples with
$\Psi^b_1$ as follows

\begin{equation}
\Psi^b_2=\sum_{n=1}^{N(N+1)} \frac{ \langle \psi^b_2(n) | H |
\Psi^b_1 \rangle}{\sqrt{\sum_{n=1}^{N(N+1)} \langle \psi^b_2(n) | H
| \Psi^b_1 \rangle^2}} |\psi^b_2(n)\rangle \equiv
\sum_{n=1}^{N(N+1)} b_n |\psi^b_2(n)\rangle .
\end{equation}

We then construct an orthonormal basis in the 2 phonons subspace,
with states $\Phi^b_2=\sum_n a_n \psi^b_2(n)$ such that

\begin{equation}
\langle \Phi^b_2 | \Psi^b_2 \rangle = \sum_{n=1}^{N(N+1)} a_n b_n  =
0.
\end{equation}

We finally find that

\begin{eqnarray}
\langle \Psi^b_2 | H | \Psi^b_1 \rangle &=& \sum_{n=1}^{N(N+1)} b_n
\langle \psi^b_2(n) | H | \Psi^b_1 \rangle \\ \nonumber &=&
\frac{1}{\sqrt{\sum_{n=1}^{N(N+1)}  \langle \psi^b_2(n) | H |
\Psi^b_1 \rangle^2}} \sum_{n=1}^{N(N+1)} \langle \psi^b_2(n) | H |
\Psi^b_1 \rangle^2
\\ \nonumber
&=& \sqrt{\sum_{n=1}^{N(N+1)}  \langle \psi^b_2(n) | H | \Psi^b_1
\rangle^2} > 0,
\end{eqnarray}

\begin{equation}
\langle \Phi^b_2 |H| \Psi^b_1 \rangle = 0,
\end{equation}

and

\begin{eqnarray}\label{Psib2_Phib1}
\langle \Psi^b_2 |H| \Phi^b_1 \rangle &=& \sum_{n=1}^{N(N+1)}
\sum_{i=1}^{N} b_n c_i \langle \psi^b_2(n) | H | \psi^b_1(i) \rangle
\\ \nonumber &=& \frac{1}{\sqrt{N}\sqrt{\sum_{n=1}^{N(N+1)} \langle
\psi^b_2(n) | H | \Psi^b_1 \rangle^2}} \sum_{n=1}^{N(N+1)}
\sum_{i=1}^{N} \sum_{k=1}^{N} c_i \langle \psi^b_2(n) | H |
\psi^b_1(i) \rangle \langle \psi^b_2(n) | H | \psi^b_1(k) \rangle
\\ \nonumber &\propto& \sum_{i=1}^{N}
\sum_{k=1}^{N} M^{(2)}_{ik} c_i = \sum_{i=1}^{N} c_i \left[ A +
B(N-1) \right] = 0,
\end{eqnarray}

where we made used the fact that $\sum_i c_i = 0$ and the relation

\begin{equation}\label{AB_relation}
M^{(2)}_{ik}\equiv \sum_{n=1}^{N(N+1)} \langle \psi^b_1(i)  | H |
\psi^b_2(n) \rangle \langle \psi^b_2(n) | H | \psi^b_1(k) \rangle =
A \delta_{ik} + B (1-\delta_{ik}).
\end{equation}

For this simplified model I could compute $\langle \Psi^b_2 |H|
\Psi^b_1 \rangle$, A and B. In particular I found that A and B do
not depend on the indexes $i,k$.

We now turn to the \textbf{3 phonons subspace}. This subspace is
spanned by $N^b_3=\frac{1}{3}N(N+1)(2N+1)$ states $\psi^b_3(i,
P)\equiv \psi^b_3(n)$, where we label them with a single index
$n=1,...,N^b_3$ for clarity. Again we can construct the single 3
phonons state that couples with $\Psi^b_2$ as follows

\begin{equation}
\Psi^b_3=\sum_{n=1}^{N^b_3} \frac{ \langle \psi^b_3(n) | H |
\Psi^b_2 \rangle}{\sqrt{\sum_{n=1}^{N^b_3} \langle \psi^b_3(n) | H |
\Psi^b_2 \rangle^2}} |\psi^b_3(n)\rangle \equiv \sum_{n=1}^{N^b_3}
d_n |\psi^b_3(n)\rangle .
\end{equation}

We then construct an orthonormal basis in the 3 phonons subspace,
with states $\Phi^b_3=\sum_n e_n \psi^b_3(n)$ such that

\begin{equation}
\langle \Phi^b_3 | \Psi^b_3 \rangle = \sum_{n=1}^{N^b_3} d_n e_n  =
0.
\end{equation}

We finally find that

\begin{eqnarray}
\langle \Psi^b_3 | H | \Psi^b_2 \rangle &=& \sum_{n=1}^{N^b_3} d_n
\langle \psi^b_3(n) | H | \Psi^b_2 \rangle \\ \nonumber &=&
\frac{1}{\sqrt{\sum_{n=1}^{N^b_3}  \langle \psi^b_3(n) | H |
\Psi^b_2 \rangle^2}} \sum_{n=1}^{N^b_3} \langle \psi^b_3(n) | H |
\Psi^b_2 \rangle^2
\\ \nonumber
&=& \sqrt{\sum_{n=1}^{N^b_3}  \langle \psi^b_3(n) | H | \Psi^b_2
\rangle^2} > 0,
\end{eqnarray}

\begin{equation}
\langle \Phi^b_3 |H| \Psi^b_2 \rangle = 0,
\end{equation}

and

\begin{eqnarray}\label{Psib2_Phib1}
\langle \Psi^b_3 |H| \Phi^b_2 \rangle &=& \sum_{n=1}^{N^b_3}
\sum_{i=1}^{N(N+1)} d_n a_i \langle \psi^b_3(n) | H | \psi^b_2(i)
\rangle
\\ \nonumber &\propto& \sum_{n=1}^{N^b_3}
 \sum_{i=1}^{N(N+1)} a_i \langle \psi^b_3(n) | H
| \Psi^b_2 \rangle \langle \psi^b_3(n) | H | \psi^b_2(i) \rangle
\\ \nonumber &\propto&  \sum_{n=1}^{N^b_3}
 \sum_{i=1}^{N(N+1)} \sum_{j=1}^{N(N+1)} a_i b_j \langle \psi^b_3(n) | H | \psi^b_2(i) \rangle
\langle \psi^b_3(n) | H | \psi^b_2(j) \rangle
\\ \nonumber &=&
 \sum_{i=1}^{N(N+1)} \sum_{j=1}^{N(N+1)} M^{(3)}_{ij} a_i b_j
\end{eqnarray}

where

\begin{equation}\label{AB_relation}
M^{(3)}_{ij} \equiv \sum_{n=1}^{N^b_3}
 \langle \psi^b_2(i) | H | \psi^b_3(n)  \rangle
\langle \psi^b_3(n) | H | \psi^b_2(j) \rangle.
\end{equation}

When $i=j$ then there are $\propto N$ states $\psi^b_3(n)$ that
couple to the 2-phonon states $\psi^b_2(i)$; when $i\neq j$ only
2-phonon states which have at least one wave vector in common can be
coupled and even in that case there is only one 3-phonon state
available for the coupling. This means that as $N\rightarrow\infty$
the ratio between diagonal and off-diagonal elements of the matrix
$M^{(3)}_{ij}$ scale as N. Moreover we find that
$\lim_{N\rightarrow\infty} M^{(3)}_{ij}=C\delta_{ij}$ with C
constant. It follows that in that limit

\begin{eqnarray}
\lim_{N\rightarrow\infty}\langle \Psi^b_3 |H| \Phi^b_2 \rangle = C
\sum_{i=1}^{N(N+1)}  a_i b_j = 0.
\end{eqnarray}

From the numerical results I found that for N=9 the approximate
matrices (\ref{Matrix_b}) and (\ref{Matrix_ac}) give the correct
position of the main peaks, but there are also two additional peaks
with intensity 100 times smaller. From the above analysis I expect
the additional peaks to disappear as N gets larger.

\subsection{Numerical results}

I performed numerical calculations to test the program, the model
and the approximation. First I could finally reproduce your
aggregate with 4x4 molecules, $J_c=0$, 1 molecule per cell and the
exact treatment of the $\bfq=0$ phonon mode. The results are shown
in Fig. \ref{Fig_Spano_4x4_3P}.

\begin{figure}
\begin{center}
  \includegraphics[width=12cm]{Spano_4x4_3P.PDF}
  \caption{}  \label{Fig_Spano_4x4_3P}
\end{center}
\end{figure}

Such results can be compared with the spectra of a 3x3 cells (9
molecules) aggregate of my model, obtained by also treating the
$\bfq=0$ phonon mode exactly (fig. \ref{Fig_Spano_b0_N3}). Notice
that I finally got $\beta<\hbar\Omega$, but the results are much
different. I had not time to investigate in detail the origin of the
discrepancy but it must be due to the different "cut" of the
aggregate and the different boundary conditions imposed.

\begin{figure}
\begin{center}
  \includegraphics[width=12cm]{Spano_b0_N3.PDF}
  \caption{}\label{Fig_Spano_b0_N3}
\end{center}
\end{figure}

In order to explore the large N limit I also computed the spectra
for a 3x3 and a 7x7 cells aggregate, including up to 2 phonon
states, with a $J_c=0$ band dispersion but with the $\bfq=0$ phonon
mode not treated exactly. Notice that the different treatment of the
$\bfq=0$ phonon mode does not produce a big effect. The most
important conclusion that can be drawn from the comparison of the
3x3 and 7x7 (2 phonons) spectra is that the infinite crystal limit
is approached very quickly and that the splitting seems to survive
that limit. This can be also seen from the analytical treatment of
the flat band model.

\begin{figure}
\begin{center}
  \includegraphics[width=12cm]{Spano_N3_3P.PDF}
  \caption{}\label{Fig_Spano_N3_3P}
\end{center}
\end{figure}

\begin{figure}
\begin{center}
  \includegraphics[width=12cm]{Spano_N7_2P.PDF}
  \caption{}\label{Fig_Spano_N7_2P}
\end{center}
\end{figure}

I also tested the flat band model. Assuming two completely flat
bands separated by about 1 eV, I computed the exact spectrum and
compared it with the spectrum obtained with the approximate
analytical model, i.e. by neglecting states built on the upper band
excitons.

\begin{figure}
\begin{center}
  \includegraphics[width=12cm]{Flat_band.PDF}
  \caption{}\label{Fig_Flat_band}
\end{center}
\end{figure}

Finally I computed the spectrum of a 3x3 cells aggregate with a
realistic band dispersion and the splitting nearly vanished. This
seems to suggest that the band dispersion must be flattened by the
renormalization due to higher energy phonon modes.

\begin{figure}
\begin{center}
  \includegraphics[width=12cm]{PS_bands.PDF}
  \caption{}\label{Fig_PS_bands}
\end{center}
\end{figure}

\section{Emission}

\subsection{General definitions}

Emission polarized along direction $\bfu$ is given by

\beq S_\bfu(\hbar\omega)=R(T)
\frac{1}{Z}\sum_{\bfk,dc,n}e^{-\frac{E_{\bfk,dc,n}-E_{LBE}}{k_BT}}
S_\bfu^{(\bfk,dc,n)}(\hbar\omega), \eeq

where the partition function is

\beq Z\equiv\sum_{\bfk,dc,n}e^{-\frac{E_{\bfk,dc,n}-E_{LBE}}{k_BT}}
\eeq

\beq S_\bfu^{(\bfk,dc,n)}(\hbar\omega)=\sum_{\nu_t=0,1,...}
I_\bfu^{0-\nu_t}(\bfk,dc,n)\left[\frac{E_{\bfk,dc,n}-\nu_t\hbar\omega_0}{E_{LBE}}\right]^3
\frac{W_e(\hbar\omega-E_{\bfk,dc,n}+\nu_t\hbar\omega_0)}{W_e(0)}
\eeq

\beq
I_\bfu^{0-\nu_t}(\bfk,dc,n)=\sum_{\{\nu_{\bfn,\alpha}\}}{}'\left|\langle\psi_{\bfk,dc,n}|\hat{\mathbf{D}}\cdot
\mathbf{u} \prod_{\bfn,\alpha}|g_{\bfn,\alpha},\nu_{\bfn,\alpha}
\rangle\right|^2, \eeq



\beq \hat{\mathbf{D}}=\sum_{\bfn,\alpha} \mathbf{d}_{\alpha}
|\bfn,\alpha\rangle\langle g | + h.c.\eeq

\beq R(T)=\frac{P}{\gamma_r+\gamma_{nr}(T)} \eeq



%%%%%%%%%%%%%%%%%%%%%%%%%%%%%%%%%%%%%%%%%%%%%%%
\subsection{Weak electronic coupling: two-particle approximation}

Conditions: $W\ll\hbar\omega_0$ and $k_B T \ll \hbar\omega_0$. We
assume that the emitting states are only the $\sigma$ lowest energy
 states in each Davydov subspace. We now consider a multi particle basis set containing one- and two-particle states. The eigenstates are written as  

\beq \psi_{\bfk,dc,n=0}=\sum_{\alpha,\umu} c_{\alpha,\umu}(\bfk,dc)
|\bfk,\alpha,\umu\rangle=\sum_{\alpha,\umu,\bfn}
c_{\alpha,\umu}(\bfk,dc)  \frac{e^{-i\bfk
\mathbf{\rho}_\alpha}}{\sqrt{N}} e^{i \bfk \bfn} | \bfn,\alpha,\umu
\rangle  \eeq

\begin{equation}
|\bfk,\alpha,\umu\rangle \equiv \frac{e^{-i\bfk
\mathbf{\rho}_\alpha}}{\sqrt{N}} \sum_{\bfn} e^{i \bfk \bfn} |
\bfn,\alpha,\umu \rangle .
\end{equation}

Following Spano we write

\bea
I_\bfu^{0-\nu_t}(\bfk,dc,n)&=&\delta_{\nu_t,0}I_\bfu^{(0)}(\bfk,dc,n)+\left(1-\delta_{\nu_t,0}\right)\left\{ I_\bfu^{(1)}(\nu_t;\bfk,dc,n) \right. \\
&+& \left.  \sum_{\nu_1=1}^{\nu_t-1}  I_\bfu^{(2)}(\nu_1,\nu_t-1;\bfk,dc,n) \right\} ,
\eea

where $I_\bfu^{(p)}$ is the line strength corresponding to emission to ground electronic states with a total of $\nu_t$ quanta spread over $p$ molecules.
The 0-0 emission line only contains terms due to one-particle states

\begin{eqnarray}
I_\bfu^{(0)}(\bfk,dc,n)&=&\left|\langle\psi_{\bfk,dc,n}|\hat{\mathbf{D}}\cdot
\mathbf{u} |0
\rangle\right|^2=\left|\langle\psi_{\bfk,dc,n}|\hat{\mathbf{D}}
|0\rangle  \cdot
\mathbf{u}\right|^2\\
&=&\left|\sum_{\alpha,\tmu,\bfn}\sum_{\bfm,\beta}
c_{\alpha,\tmu}(\bfk,dc,n) \frac{e^{-i\bfk
\mathbf{\rho}_\alpha}}{\sqrt{N}} e^{i \bfk \bfn} \langle
\bfn,\alpha|\bfm,\beta\rangle_{\rm el} \langle\tmu|0\rangle_{\rm ph} \bfd_\beta\cdot
\mathbf{u}\right|^2\\
&=& \left|\sum_{\alpha,\tmu,\bfn} c_{\alpha,\tmu}(\bfk,dc,n)
\frac{e^{-i\bfk \mathbf{\rho}_\alpha}}{\sqrt{N}} e^{i \bfk \bfn}
S_{\tmu0} \bfd_\alpha\cdot \mathbf{u}\right|^2\\
&=&\delta(\bfk)\left|\sum_{\alpha,\tmu} c_{\alpha,\tmu}(\bfk,dc,n)
\sqrt{N} e^{-i\bfk \mathbf{\rho}_\alpha}
S_{\tmu0} \bfd_\alpha\cdot \mathbf{u}\right|^2\\
&=&N\delta(\bfk)\left|\sum_{\alpha,\tmu} c_{\alpha,\tmu}(\bfk=0,dc,n)
S_{\tmu0} \bfd_\alpha\cdot \mathbf{u}\right|^2\\
\end{eqnarray}

where we used the relation

\beq \sum_\bfn \frac{1}{N} e^{i \bfk \bfn}=\delta (\bfk). \eeq

Now we give the expression for $I_\bfu^{(1)}(\nu_t;\bfk,dc,n)$, which consider emission ending on states $|g;\nu_t,\bfl,\gamma\rangle\equiv |g\rangle_{\rm el}|{\nu_{t }}_{\bfl,\gamma}\rangle_{\rm ph}$ in which all molecules are in their electronic ground state and there are $\nu_t$  vibrational quanta on molecule $(\bfl,\gamma)$. Here we have an additional contribution due to two-particle states. The phonon cloud in this case is indicated for convenience as $\umu\equiv \{ \tmu;\nu,\bf{p},\sigma\}$, where $\tmu$ indicates the number of phonons on the (electronically) excited molecule and $\nu$ indicates the number of phonons in the (electronic) ground state molecule residing at the relative position $(\bf{p},\sigma)$ with respect to the (electronically) excited molecule.

\begin{eqnarray}
&&I_\bfu^{(1)}(\nu_t;\bfk,dc,n) =  \\ \nonumber
&=& \sum_{\bfl,\gamma}\left|\langle\psi_{\bfk,dc,n}|\hat{\mathbf{D}}\cdot
\mathbf{u} |g;\bfl,\gamma,\nu_t\rangle\right|^2\\ \nonumber
&=&\sum_{\bfl,\gamma}\left|\sum_{\bfn,\alpha,\tmu,\nu,\bfp,\sigma}\sum_{\bfm,\beta}
c_{\alpha,\umu}(\bfk,dc,n) \frac{e^{-i\bfk
\mathbf{\rho}_\alpha}}{\sqrt{N}} e^{i \bfk \bfn} \langle
\bfn,\alpha|\bfm,\beta\rangle_{\rm el}
\langle\bfn,\alpha,\tmu;\bfn+\bfp,\sigma,\nu|\bfl,\gamma,\nu_t\rangle_{\rm ph} \bfd_\beta\cdot
\mathbf{u}\right|^2
%
\\ \nonumber
&=&\sum_{\bfl,\gamma}\left|\sum_{\tmu}
c_{\gamma,\tmu}(\bfk,dc,n) \frac{e^{-i\bfk
\mathbf{\rho}_\gamma}}{\sqrt{N}} e^{i \bfk \bfl}S_{\tmu\nu_t} \bfd_\gamma\cdot
\mathbf{u}+ \sum_{\bfp,\alpha,\tmu}
c_{\alpha,\{\tmu,\bfp,\gamma,\nu_t\}}(\bfk,dc,n) \frac{e^{-i\bfk
\mathbf{\rho}_\alpha}}{\sqrt{N}} e^{i \bfk \bfl}e^{-i \bfk \bfp} S_{\tmu 0} \bfd_\alpha\cdot
\mathbf{u}\right|^2 \\ \nonumber
%
&=&\sum_{\gamma}\left|\sum_{\tmu}
c_{\gamma,\tmu}(\bfk,dc,n) e^{-i\bfk
\mathbf{\rho}_\gamma} S_{\tmu\nu_t} \bfd_\gamma\cdot
\mathbf{u}+ \sum_{\bfp,\alpha,\tmu}
c_{\alpha,\{\tmu,\bfp,\gamma,\nu_t\}}(\bfk,dc,n) e^{-i\bfk
\mathbf{\rho}_\alpha}e^{-i \bfk \bfp} S_{\tmu 0} \bfd_\alpha\cdot
\mathbf{u}\right|^2
\end{eqnarray}

where the first term is due to one-particle states and the second to two-particle states.  We made use of the relation

\beq
\langle\bfn,\alpha,\tmu;\bfn+\bfp,\sigma,\nu|\bfl,\gamma,\nu_t\rangle_{\rm ph}=\delta_{\bfn,\bfl} \delta_{\alpha\gamma} \delta_{\nu 0} S_{\tmu\nu_t}+\delta_{\bfn+\bfp,\bfl} \delta_{\sigma\gamma} \delta_{\nu \nu_t} S_{\tmu 0} (1-\delta_{\nu 0} )
\eeq

\begin{eqnarray}
&&I_\bfu^{(2)}(\nu_1,\nu_2;\bfk,dc,n) =  \\ \nonumber
&=& \sum_{\bfl,\gamma,\bfj,\xi}\left|\langle\psi_{\bfk,dc,n}|\hat{\mathbf{D}}\cdot
\mathbf{u} |g;\bfl,\gamma,\nu_1;\bfj,\xi,\nu_2\rangle\right|^2\\ \nonumber
%
&=&\sum_{\bfl,\gamma,\bfj,\xi}\left|\sum_{\bfn,\alpha,\tmu,\nu,\bfp,\sigma}\sum_{\bfm,\beta}
c_{\alpha,\umu}(\bfk,dc,n) \frac{e^{-i\bfk
\mathbf{\rho}_\alpha}}{\sqrt{N}} e^{i \bfk \bfn} \langle
\bfn,\alpha|\bfm,\beta\rangle_{\rm el}
\langle\bfn,\alpha,\tmu;\bfn+\bfp,\sigma,\nu|\bfl,\gamma,\nu_1;\bfj,\xi,\nu_2\rangle_{\rm ph} \bfd_\beta\cdot
\mathbf{u}\right|^2
%
\\ \nonumber
&=&\sum_{\bfl,\gamma,\bfj,\xi}\left| \sum_{\tmu}
c_{\gamma,\{\tmu,\bfj-\bfl,\xi,\nu_2\}}(\bfk,dc,n) \frac{e^{-i\bfk
\mathbf{\rho}_\gamma}}{\sqrt{N}} e^{i \bfk \bfl}S_{\tmu\nu_1} \bfd_\gamma\cdot
\mathbf{u}+ 
\sum_{\tmu}
c_{\xi,\{\tmu,\bfl-\bfj,\gamma,\nu_1\}}(\bfk,dc,n) \frac{e^{-i\bfk
\mathbf{\rho}_\xi}}{\sqrt{N}} e^{i \bfk \bfj}S_{\tmu\nu_2} \bfd_\xi\cdot
\mathbf{u}\right|^2 \\ \nonumber
%
&=&\sum_{\gamma,\bfr,\xi}\left| \sum_{\tmu}
c_{\gamma,\{\tmu,\bfj-\bfl,\xi,\nu_2\}}(\bfk,dc,n) e^{-i\bfk
\mathbf{\rho}_\gamma} S_{\tmu\nu_1} \bfd_\gamma\cdot
\mathbf{u}+ 
\sum_{\tmu}
c_{\xi,\{\tmu,\bfl-\bfj,\gamma,\nu_1\}}(\bfk,dc,n) e^{-i\bfk
\mathbf{\rho}_\xi}e^{i \bfk \bfr}S_{\tmu\nu_2} \bfd_\xi\cdot
\mathbf{u}\right|^2 
\end{eqnarray}

where all terms are due to two-particle states.  We defined 
\beq
\bfr\equiv\bfj-\bfl
\eeq

and made use of the relation

\beq
\langle\bfn,\alpha,\tmu;\bfn+\bfp,\sigma,\nu|\bfl,\gamma,\nu_1;\bfj,\xi,\nu_2\rangle_{\rm ph} =\delta_{\bfn,\bfl} \delta_{\alpha\gamma} \delta_{\bfn+\bfp,\bfj} \delta_{\sigma,\xi} \delta_{\nu \nu_2} S_{\tmu\nu_1}+\delta_{\bfn,\bfj} \delta_{\alpha\xi} \delta_{\bfn+\bfp,\bfl} \delta_{\sigma,\gamma} \delta_{\nu \nu_1} S_{\tmu\nu_2}
\eeq

%%%%%%%%%%%%%%%%%%%%%%%%%%%%%%%%%%%%%%%%%%%%%%%%%%%%%%%%
\subsection{Strong electronic coupling}

Conditions: $W\gg\hbar\omega_0$ and $k_B T \ll \hbar\omega_0$. We
assume that the single emitting state is the lowest energy at each
$\bfk$ and we expand it to the first order perturbation theory. The
Hamiltonian is

\begin{eqnarray}\label{H_crystal}
H&=&H_0+H_{1},
\end{eqnarray}

where

\begin{equation}\label{H_0}
H_{0}=\hbar\omega_{0-0}+D+ \sum_{\bfk,\alpha,\beta}
\tilde{J}_{\alpha\beta}(\bfk) \left| \bfk,\alpha,\tmu \right\rangle
\left\langle \bfk,\beta,\tnu \right|+\hbar\omega_0\sum_{\bfq,\alpha}
b^\dagger_{\bfq,\alpha}b_{\bfq,\alpha}+\lambda_0^2\hbar\omega_0 ,
\end{equation}

is the unperturbed Hamiltonian, which can be easily diagonalized and

\begin{equation}\label{H_1}
H_{1}=\frac{\lambda_0\hbar\omega_0}{\sqrt{N}}\sum_{\bfk,\bfq,\alpha}
\left[ b^\dagger_{\bfq,\alpha}\left| \bfk,\alpha \right\rangle
\left\langle \bfk+\bfq,\alpha \right| + b_{\bfq,\alpha}\left|
\bfk,\alpha \right\rangle \left\langle \bfk-\bfq,\alpha \right|
\right],
\end{equation}

is the perturbation. Eigenstates of $H_0$ consist of a product of a
pure exciton part and a pure phonon part. To zero order, the lowest
energy exciton with wave vector $\bfk$ contains no phonons

\beq
\psi_{\bfk,dc=\mathrm{Lowest}}^{(0)}\equiv\psi_{\bfk}^{(0)}=|\bfk,L\rangle\otimes|0\rangle_{ph}=
\sum_{\alpha} c_{\alpha}(\bfk,L)
|\bfk,\alpha\rangle\otimes|0\rangle_{ph}=\sum_{\alpha,\bfn}
c_{\alpha}(\bfk,L)  \frac{e^{-i\bfk \mathbf{\rho}_\alpha}}{\sqrt{N}}
e^{i \bfk \bfn} | \bfn,\alpha \rangle\otimes|0\rangle_{ph},  \eeq

where $|0\rangle_{ph}$ is the vacuum phonon state and

\begin{equation}\label{exc_state_def}
|\bfk,\alpha\rangle \equiv \frac{e^{-i\bfk
\mathbf{\rho}_\alpha}}{\sqrt{N}} \sum_{\bfn} e^{i \bfk \bfn} |
\bfn,\alpha \rangle.
\end{equation}

Perturbation couples $\psi_{\bfk}^{(0)}$ with all the one phonon
states with total wave vector $\bfk$

\bea \psi_{\bfq,dc,\beta}^{1ph,\bfk}&=&|\bfq,dc\rangle\otimes
b^\dagger_{\bfk-\bfq,\beta}|0\rangle_{ph}= \sum_{\alpha}
c_{\alpha}(\bfq,dc) |\bfq,\alpha\rangle\otimes
b^\dagger_{\bfk-\bfq,\beta}|0\rangle_{ph}\\
&=&\sum_{\alpha,\bfn} c_{\alpha}(\bfq,dc) \frac{e^{-i\bfq
\mathbf{\rho}_\alpha}}{\sqrt{N}} e^{i \bfq \bfn} | \bfn,\alpha
\rangle\otimes b^\dagger_{\bfk-\bfq,\beta}|0\rangle_{ph} \\
&=&\sum_{\alpha,\bfn} c_{\alpha}(\bfq,dc) \frac{e^{-i\bfq
\mathbf{\rho}_\alpha}}{\sqrt{N}} e^{i \bfq \bfn} | \bfn,\alpha
\rangle\otimes \sum_{\bfm}\frac{e^{i (\bfk-\bfq) \bfm}}{\sqrt{N}}
b^\dagger_{\bfm,\beta}|0\rangle_{ph}. \eea

\bea &&\langle \psi_{\bfk}^{(0)} | H_1 |
\psi_{\bfq,dc,\beta}^{1ph,\bfk} \rangle = \sum_{\alpha,\gamma}
c^*_{\alpha}(\bfk,L) c_{\gamma}(\bfq,dc) \langle
\bfk,\alpha|\otimes\langle 0|_{ph} H_1  | \bfq,\gamma \rangle\otimes
b^\dagger_{\bfk-\bfq,\beta}|0\rangle_{ph} \\
&=& \frac{\lambda_0\hbar\omega_0}{\sqrt{N}}\sum_{\gamma}
c^*_{\beta}(\bfk,L) c_{\gamma}(\bfq,dc) \langle
\bfk,\beta|\otimes\langle 0|_{ph} \left[
\sum_{\bfk',\bfq',\alpha'}b_{\bfq',\alpha'}\left| \bfk',\alpha'
\right\rangle \left\langle \bfk'-\bfq',\alpha' \right|\right]  |
\bfq,\gamma \rangle\otimes b^\dagger_{\bfk-\bfq,\beta}|0\rangle_{ph}
\\
&=& \frac{\lambda_0\hbar\omega_0}{\sqrt{N}} c^*_{\beta}(\bfk,L)
c_{\beta}(\bfq,dc)
 \eea

\bea \psi_{\bfk}^{(1)}&=&\sum_{\bfq,dc,\beta} \frac{\langle
\psi_{\bfq,dc,\beta}^{1ph,\bfk}  | H_1 | \psi_{\bfk}^{(0)}
\rangle}{J_{\bfk,L}-(J_{\bfq,dc}+\hbar\omega_0)} |
\psi_{\bfq,dc,\beta}^{1ph,\bfk} \rangle \\
&=&\frac{\lambda_0\hbar\omega_0}{\sqrt{N}} \sum_{\bfq,dc,\beta}
\frac{c_{\beta}(\bfk,L)
c^*_{\beta}(\bfq,dc)}{J_{\bfk,L}-(J_{\bfq,dc}+\hbar\omega_0)} |
\psi_{\bfq,dc,\beta}^{1ph,\bfk} \rangle\\
&=&\frac{\lambda_0\hbar\omega_0}{\sqrt{N}} \sum_{\bfq,dc,\beta}
\frac{c_{\beta}(\bfk,L)
c^*_{\beta}(\bfq,dc)}{J_{\bfk,L}-(J_{\bfq,dc}+\hbar\omega_0)}
|\bfq,dc\rangle\otimes b^\dagger_{\bfk-\bfq,\beta}|0\rangle_{ph}
\eea

To the first order the emitting state is therefore

\beq \psi_{\bfk}=\psi_{\bfk}^{(0)}+\psi_{\bfk}^{(1)}. \eeq

Emission to the ground state can only proceed from
$\psi_{\bfk}^{(0)}$ so we can write

\begin{eqnarray}
I_\bfu^{0-0}(\bfk)&=&\left|\langle\psi_{\bfk}|\hat{\mathbf{D}}\cdot
\mathbf{u} |0
\rangle\right|^2=\left|\langle\psi_{\bfk}^{(0)}|\hat{\mathbf{D}}
|0\rangle \cdot
\mathbf{u}\right|^2\\
&=& \left|\langle \bfk,L|\hat{\mathbf{D}} |0\rangle_{el} \cdot
\mathbf{u}\right|^2=\delta(\bfk)\left|\langle
\bfk=0,L|\hat{\mathbf{D}}
|0\rangle_{el} \cdot \mathbf{u}\right|^2\\
&=&\delta(\bfk)\left|\sum_{\alpha,\bfn} c_{\alpha}(\bfk=0,L)
\frac{1}{\sqrt{N}} \langle \bfn,\alpha |\hat{\mathbf{D}}
|0\rangle_{el} \cdot
\mathbf{u}\right|^2=\delta(\bfk)\left|\sum_{\alpha,\bfn}
c_{\alpha}(\bfk=0,L) \frac{1}{\sqrt{N}}
\mathbf{d}_\alpha \cdot \mathbf{u}\right|^2 \\
&=& N \delta(\bfk) \left|\sum_{\alpha} c_{\alpha}(0,L)
\mathbf{d}_\alpha \cdot \mathbf{u}\right|^2 = N \delta(\bfk) \left|
\mathbf{d}_L \cdot \mathbf{u}\right|^2, \eea

where we defined

\beq \mathbf{d}_{dc}\equiv\sum_{\alpha} c_{\alpha}(0,dc)
\mathbf{d}_\alpha. \eeq

Emission to one phonon states can only proceed from
$\psi_{\bfk}^{(1)}$ so we can write

\begin{eqnarray}
I_\bfu^{0-1}(\bfk)&=&\sum_{\bfl,\gamma}\left|\langle g;\bfl,\gamma
|\hat{\mathbf{D}}\cdot \mathbf{u}
|\psi_{\bfk}^{(1)}\rangle\right|^2\\
&=& \frac{\lambda_0^2(\hbar\omega_0)^2}{N}\sum_{\bfl,\gamma}\left|
\sum_{\bfq,dc,\beta} \frac{c_{\beta}(\bfk,L)
c^*_{\beta}(\bfq,dc)}{J_{\bfk,L}-(J_{\bfq,dc}+\hbar\omega_0)}
\langle 0 |\hat{\mathbf{D}}\cdot \mathbf{u} |\bfq,dc\rangle_{el}
\langle g;\bfl,\gamma | b^\dagger_{\bfk-\bfq,\beta}|0\rangle_{ph}
\right|^2 \\
&=& \frac{\lambda_0^2(\hbar\omega_0)^2}{N}\sum_{\bfl,\gamma}\left|
\sum_{\bfq,dc,\beta} \frac{c_{\beta}(\bfk,L)
c^*_{\beta}(\bfq,dc)}{J_{\bfk,L}-(J_{\bfq,dc}+\hbar\omega_0)}
\langle 0 |\hat{\mathbf{D}}\cdot \mathbf{u} |\bfq,dc\rangle_{el}
\delta_{\gamma\beta}\frac{e^{i(\bfk-\bfq)\bfl}}{\sqrt{N}} \right|^2\\
&=& \frac{\lambda_0^2(\hbar\omega_0)^2}{N}\sum_{\bfl,\gamma}\left|
\sum_{dc} \frac{c_{\gamma}(\bfk,L)
c^*_{\gamma}(\bfq=0,dc)}{J_{\bfk,L}-(J_{\bfq=0,dc}+\hbar\omega_0)}
\langle 0 |\hat{\mathbf{D}}\cdot \mathbf{u} |\bfq=0,dc\rangle_{el}
\frac{e^{i\bfk\bfl}}{\sqrt{N}} \right|^2 \\
&=& \lambda_0^2(\hbar\omega_0)^2\sum_{\gamma}\left| \sum_{dc}
\frac{c_{\gamma}(\bfk,L)
c^*_{\gamma}(0,dc)}{J_{\bfk,L}-(J_{0,dc}+\hbar\omega_0)} \langle 0
|\hat{\mathbf{D}}\cdot \mathbf{u} |0,dc\rangle_{el}
\frac{1}{\sqrt{N}} \right|^2\\
&=& \lambda_0^2(\hbar\omega_0)^2\sum_{\gamma}\left| \sum_{dc}
\frac{c_{\gamma}(\bfk,L)
c^*_{\gamma}(0,dc)}{J_{\bfk,L}-(J_{0,dc}+\hbar\omega_0)}
 \mathbf{d}_{dc} \cdot \mathbf{u} \right|^2.\eea

In the above derivation we made use of the relations

\beq \langle g;\bfl,\gamma
|b^\dagger_{\bfk-\bfq,\beta}|0\rangle_{ph}=\delta_{\gamma\beta}\frac{e^{i(\bfk-\bfq)\bfl}}{\sqrt{N}}.
\eeq

\bea \langle 0 |\hat{\mathbf{D}}\cdot \mathbf{u}
|0,dc\rangle_{el}&=&\langle 0 |\hat{\mathbf{D}}\cdot \mathbf{u}
\sum_{\alpha,\bfn} c_{\alpha}(0,dc) \frac{1}{\sqrt{N}}| \bfn,\alpha
\rangle_{el}\\
&=&\sqrt{N}  \sum_{\alpha} c_{\alpha}(0,dc) \mathbf{d}_\alpha \cdot
\mathbf{u}=\sqrt{N}\mathbf{d}_{dc} \cdot \mathbf{u}   \eea


\section*{Appendix A. Harmonic oscillator}

Hamiltonian

\begin{equation}\label{Harmonic_oscillator}
H=\frac{p^2}{2m}+\frac{1}{2}m\omega^2 x^2
\end{equation}

Eigenfunctions

\begin{equation}\label{Harmonic_oscillator_eigenfunctions}
\psi_n (x)=\frac{1}{\sqrt{2^n n!}} \left(
\frac{m\omega}{\pi\hbar}\right)^{1/4} \exp \left( -\frac{m\omega
x^2}{2\hbar} \right)H_n \left( \sqrt{\frac{m\omega}{\hbar}} x
\right)
\end{equation}

with eigenenergies $E_n=\hbar\omega (n+1/2)$. We can introduce
creation and destruction operators

\begin{eqnarray}\label{Harmonic_oscillator_operators}
a=\sqrt{\frac{m\omega}{2\hbar}}\left( x+\frac{i}{m\omega}p\right) \\
a^\dagger=\sqrt{\frac{m\omega}{2\hbar}}\left( x -
\frac{i}{m\omega}p\right)
\end{eqnarray}

which operate on the eigenfunctions as


\begin{eqnarray}
  a| \psi_n \rangle &=& \sqrt{n}|\psi_{n-1}\rangle \\
  a^\dagger| \psi_n \rangle &=& \sqrt{n+1}|\psi_{n+1}\rangle
\end{eqnarray}

so that Hamiltonian reads $H=\hbar\omega (a^\dagger a +1/2)$ with $[
a, a^\dagger]=1$. If we measure energy in units of $\hbar\omega$ and
lengths in units of $\sqrt{\frac{2\hbar}{m\omega}}$ the Hamiltonian
reads

\begin{equation}\label{Harmonic_oscillator}
H=-\frac{1}{4}\frac{\partial^2}{\partial Q^2}+Q^2
\end{equation}

and eigenfunctions become

\begin{equation}\label{Harmonic_oscillator_eigenfunctions}
\psi_n (Q)=\frac{1}{\sqrt{2^n n!}} \left( \pi/2 \right)^{-1/4} \exp
\left( -Q^2 \right) H_n \left( \sqrt{2} Q \right).
\end{equation}

A harmonic oscillator displaced by $\lambda$ in "natural"
coordinates has Hamiltonian

\begin{equation}\label{Harmonic_oscillator}
H=-\frac{1}{4}\frac{\partial^2}{\partial Q^2}+(Q-\lambda)^2
\end{equation}

and eigenfunctions

\begin{equation}\label{Harmonic_oscillator_eigenfunctions}
\psi_{n,\lambda} (Q)=\frac{1}{\sqrt{2^n n!}} \left( \pi/2
\right)^{-1/4} \exp \left( -(Q-\lambda)^2 \right) H_n \left(
\sqrt{2} (Q-\lambda) \right).
\end{equation}

Overlaps between these functions are usually defined as
\begin{equation}\label{S_def}
S_{\mu \nu}=\langle \psi_{\mu,0} | \psi_{\nu,\lambda} \rangle
\end{equation}

Franck-Condon factors are then given by
\begin{equation}\label{FCF}
S_{0 \nu}^2=\langle \psi_{0,0} | \psi_{\nu,\lambda}
\rangle^2=\frac{\exp(-\lambda^2)\lambda^{2\nu}}{\nu!}
\end{equation}

A Huang-Rhys factor has also been defined as $S=\lambda^2$.

\section*{Appendix B. Ewald summation for sub-dipoles}

The position of a sub-dipole is given by $\bfr_{\bfn,\alpha}^i$
where $\bfn$ denotes the unit cell, $\alpha$ the molecule and $i$
the sub-dipole. In particular we can write
$\bfr_{\bfn,\alpha}^i=\bfn+\mathbf{\rho}_\alpha+\delta\mathbf{\rho}_\alpha^i$,
and we also introduce the symbols
$\delta\bfr=\delta\mathbf{\rho}_\alpha^i-\delta\mathbf{\rho}_\beta^j$
and $\bfR=\bfr_{\mathbf{0},\alpha}^i-\bfr_{\bfn,\beta}^j$. We now
consider the value of $\tilde{L}_{\alpha,\beta}(\bfk)$ and we obtain

\begin{eqnarray}
\tilde{L}_{\alpha,\beta}(\bfk)=\sum_{\bfn}{}' \sum_{i,j=1}^{N_s}
M_{\mathbf{0}\alpha i,\bfn\beta j} e^{i\bfk \bfr_{\bfn}} \equiv
\frac{1}{N_s^2} \sum_{i,j=1}^{N_s}(A+B+C)
\end{eqnarray}
where $N_s$ is the number of sub-dipoles, $M_{\mathbf{0}\alpha
i,\bfn\beta j}$ is the dipole-dipole interaction between sub-dipoles
and $\bfr_\bfn=\bfn+{\mathbf{\rho}}_\alpha-{ \mathbf{\rho}}_\beta$.
\begin{eqnarray}
A&=&\frac{4\pi}{\upsilon} \sum_\bfK
\frac{\left[\left(\bfK+\bfk\right)\bfd_\alpha\right]\left[\left(\bfK+\bfk\right)\bfd_\beta\right]}{\left(\bfK+\bfk\right)^2}
\exp\left[-\frac{\left(\bfK+\bfk\right)^2}{4 \eta^2}\right] \\
\nonumber &\times& \exp\left[-i \bfK
\left(\mathbf{\rho}_\alpha-\mathbf{\rho}_\beta\right)\right]
\exp\left[-i \left(\bfK +\bfk\right)\delta\bfr\right]
\end{eqnarray}
where $\bfd$ is the full dipole of each molecule.
\begin{eqnarray}
B&=&  \sum_\bfn {}' e^{i\bfk \bfr_{\bfn}} \left\{ \left[
\frac{\bfd_\alpha \bfd_\beta - 3 (\bfd_\alpha \hat{\bfR})(\bfd_\beta
\hat{\bfR})} {R^3}\right]\right. \left[\mathrm{erfc}(\eta
R)+\frac{2\eta
R}{\sqrt{\pi}}\exp\left(-\eta^2R^2\right)\right] \nonumber \\
\nonumber &-& \left. 3(\bfd_\alpha \hat{\bfR})(\bfd_\beta
\hat{\bfR})
\left[\frac{4\eta^3}{3\sqrt{\pi}}\exp\left(-\eta^2R^2\right)\right]
 \right\}
\end{eqnarray}

\begin{eqnarray}
C&=& -\delta_{\alpha\beta}
 \left\{ \delta_{ij} \bfd_\alpha \bfd_\beta \frac{4\eta^3}{3\sqrt{\pi}} +(1-\delta_{ij}) \left\{\left[ \frac{\bfd_\alpha
\bfd_\beta - 3 (\bfd_\alpha \hat{\delta\bfr})(\bfd_\beta
\hat{\delta\bfr})} {\delta r^3}\right]\right. \right. \nonumber \\
\nonumber  &\times& \left. \left. \left[1-\mathrm{erfc}(\eta \delta
r)-\frac{2\eta \delta r}{\sqrt{\pi}}\exp\left(-\eta^2\delta
r^2\right)\right] +3(\bfd_\alpha \hat{\delta\bfr})(\bfd_\beta
\hat{\delta\bfr})
\left[\frac{4\eta^3}{3\sqrt{\pi}}\exp\left(-\eta^2\delta
r^2\right)\right]  \right\} \right\}
\end{eqnarray}

\section*{Appendix C. Dielectric
 tensor}

The dielectric tensor can be obtained from the computed excitonic
states from the following formula

\begin{equation}\label{epsilon}
\epsilon_{ij}(\omega)=\epsilon_\infty \delta_{ij}+
\frac{2}{\epsilon_0 \upsilon \hbar} \sum_n  \frac{ (\textbf{d}_n)_i
(\textbf{d}_n)_j \omega_n}{\omega_n^2-\omega^2-i\gamma\omega},
\end{equation}

where $\upsilon$ is the cell volume, $\hbar\omega_n$ and
$\textbf{d}_n$ are the energy and transition dipole of the $n$-th
state respectively; $\gamma$ is a damping factor which can also
depend on the frequency $\omega$. We recall that there is an extra
factor 2 due to the spin degeneracy, which is appropriate to
excitonic singlet states, which must be included in the dipole
moment value $\textbf{d}_n$. The above expression corresponds to an
oscillator strength of
\begin{equation}
f_n=\frac{2 m \omega_n \textbf{d}_n^2}{e^2\hbar},
\end{equation}

for each excitonic transition, where again the extra factor 2 must
be included in the dipole moment value $\textbf{d}_n$ and it is
appropriate to excitonic singlet states. This can be seen by
comparing the above expression for $\epsilon$ with its general form:

\begin{equation}
\epsilon=\epsilon_b + \sum_n \frac{f_n
\omega_p^2}{\omega_n^2-\omega^2},
\end{equation}

where we neglected damping, $\epsilon_b$ is a background dielectric
constant and

\begin{equation}
\omega_p^2=\frac{e^2}{\upsilon m \epsilon_0}.
\end{equation}

If dipoles are expresses in Debyes, energies in eV and lengths in
{\AA}, formula (\ref{epsilon})reduces to

\begin{equation}
\epsilon_{ij}(\omega)=\epsilon_\infty \delta_{ij}+ \frac{8\pi}{1.602
\upsilon} \sum_n  \frac{ (\textbf{d}_n)_i (\textbf{d}_n)_j
E_n}{E_n^2-E^2-i\Gamma E },
\end{equation}

where $E_n=\hbar\omega_n$, $E=\hbar\omega$ and $\Gamma=\hbar\gamma$.
We obtained the above coefficient by using the following constants
and conversions
\begin{eqnarray}
\epsilon_0&=&8.8541\times 10^{-12}C^2 m^{-2}
N^{-1} \\
1 D &=& 3.336\times 10^{-30} C m = e 10^{-20} m \\
1 eV &=& 1.602 \times 10^{-19} J \\
\hbar c &=& 1.97327 \times 10^{-7} m \cdot eV \\
m c^2 &=& 0.511 \times 10^6 eV
\end{eqnarray}

The above formulas are equivalent to the gaussian units expression
\begin{equation}
\epsilon_{ij}(\omega)=\epsilon_\infty \delta_{ij}+
\frac{8\pi}{\upsilon \hbar} \sum_n  \frac{ (\textbf{d}_n)_i
(\textbf{d}_n)_j \omega_n}{\omega_n^2-\omega^2-i\gamma\omega},
\end{equation}

where dipoles are in statC $\cdot$ cm, energies in erg, lengths in
cm and  statC = cm $\sqrt{dyn}$. We recall that gaussian units
expressions can be obtained substituting each SI charge $q$ with $q
\sqrt{4\pi \epsilon_0}$.

We finally note that the oscillator strength indicated by Gaussian
is $f_n^{(Gaussian)}=f_n/3$ and gives a dipole moment in Debyes of
\begin{equation}
|\textbf{d}_n|=\sqrt{\frac{3}{2}175.8
\frac{f_n^{(Gaussian)}}{E_n}}=16.239
\sqrt{\frac{f_n^{(Gaussian)}}{E_n}},
\end{equation}
where $E_n$ is expressed in eV.

\section{Appendix D. Weak electronic coupling: vibronic approximation}

Conditions: $W\ll\hbar\omega_0$ and $k_B T \ll \hbar\omega_0$. We
assume that the emitting states are only the $\sigma$ lowest energy
vibronic states in each Davydov subspace. Vibronic states are one-particle states that can be written as

\beq \psi_{\bfk,dc,n=0}=\sum_{\alpha,\tmu} c_{\alpha,\tmu}(\bfk,dc)
|\bfk,\alpha,\tmu\rangle=\sum_{\alpha,\tmu,\bfn}
c_{\alpha,\tmu}(\bfk,dc)  \frac{e^{-i\bfk
\mathbf{\rho}_\alpha}}{\sqrt{N}} e^{i \bfk \bfn} | \bfn,\alpha,\tmu
\rangle  \eeq

\begin{equation}\label{exc_state_def}
|\bfk,\alpha,\tmu\rangle \equiv \frac{e^{-i\bfk
\mathbf{\rho}_\alpha}}{\sqrt{N}} \sum_{\bfn} e^{i \bfk \bfn} |
\bfn,\alpha,\tmu \rangle ,
\end{equation}


\begin{eqnarray}
I_\bfu^{0-0}(\bfk,dc,0)&=&\left|\langle\psi_{\bfk,dc,0}|\hat{\mathbf{D}}\cdot
\mathbf{u} |0
\rangle\right|^2=\left|\langle\psi_{\bfk,dc,0}|\hat{\mathbf{D}}
|0\rangle  \cdot
\mathbf{u}\right|^2\\
&=&\left|\sum_{\alpha,\tmu,\bfn}\sum_{\bfm,\beta}
c_{\alpha,\tmu}(\bfk,dc) \frac{e^{-i\bfk
\mathbf{\rho}_\alpha}}{\sqrt{N}} e^{i \bfk \bfn} \langle
\bfn,\alpha|\bfm,\beta\rangle \langle\tmu|0\rangle \bfd_\beta\cdot
\mathbf{u}\right|^2\\
&=& \left|\sum_{\alpha,\tmu,\bfn} c_{\alpha,\tmu}(\bfk,dc)
\frac{e^{-i\bfk \mathbf{\rho}_\alpha}}{\sqrt{N}} e^{i \bfk \bfn}
\langle\tmu|0\rangle \bfd_\alpha\cdot \mathbf{u}\right|^2\\
&=&\delta(\bfk)\left|\sum_{\alpha,\tmu} c_{\alpha,\tmu}(\bfk,dc)
\sqrt{N} e^{-i\bfk \mathbf{\rho}_\alpha}
\langle\tmu|0\rangle \bfd_\alpha\cdot \mathbf{u}\right|^2\\
&=&N\delta(\bfk)\left|\sum_{\alpha,\tmu} c_{\alpha,\tmu}(\bfk=0,dc)
S_{\tmu 0} \bfd_\alpha\cdot \mathbf{u}\right|^2\\
\end{eqnarray}

where we used the relation

\beq \sum_\bfn \frac{1}{N} e^{i \bfk \bfn}=\delta (\bfk). \eeq

Now we give the expression for the line strength of the first
replica, which ends on states $|g;\bfl,\gamma\rangle$ in which all
molecules are in their electronic ground state and there is one
vibrational quantum on molecule $(\bfl,\gamma)$:

\begin{eqnarray}
I_\bfu^{0-1}(\bfk,dc,n=0)&=&\sum_{\bfl,\gamma}\left|\langle\psi_{\bfk,dc,0}|\hat{\mathbf{D}}\cdot
\mathbf{u} |g;\bfl,\gamma\rangle\right|^2\\
&=&\sum_{\bfl,\gamma}\left|\sum_{\alpha,\tmu,\bfn}\sum_{\bfm,\beta}
c_{\alpha,\tmu}(\bfk,dc) \frac{e^{-i\bfk
\mathbf{\rho}_\alpha}}{\sqrt{N}} e^{i \bfk \bfn} \langle
\bfn,\alpha|\bfm,\beta\rangle_{el}
\langle\tmu_{\bfn,\alpha}|\bfl,\gamma\rangle_{ph} \bfd_\beta\cdot
\mathbf{u}\right|^2\\
&=&\sum_{\bfl,\gamma}\left|\sum_{\tmu}\sum_{\bfm,\beta}
c_{\gamma,\tmu}(\bfk,dc) \frac{e^{-i\bfk
\mathbf{\rho}_\gamma}}{\sqrt{N}} e^{i \bfk \bfl} \langle
\bfl,\gamma|\bfm,\beta\rangle_{el}
\langle\tmu|1\rangle_{ph}\bfd_\beta\cdot
\mathbf{u}\right|^2\\
&=&\sum_{\bfl,\gamma}\left|\sum_{\tmu} c_{\gamma,\tmu}(\bfk,dc)
\frac{e^{-i\bfk \mathbf{\rho}_\gamma}}{\sqrt{N}} e^{i \bfk \bfl}
S_{\tmu 1}\bfd_\gamma\cdot \mathbf{u}\right|^2\\
&=&\frac{1}{N}\sum_{\bfl,\gamma}\left|\sum_{\tmu}
c_{\gamma,\tmu}(\bfk,dc) e^{-i\bfk \mathbf{\rho}_\gamma} S_{\tmu
1}\bfd_\gamma\cdot \mathbf{u}\right|^2\\
&=&\sum_{\gamma}\left|\sum_{\tmu} c_{\gamma,\tmu}(\bfk,dc) S_{\tmu
1}e^{-i\bfk \mathbf{\rho}_\gamma} \bfd_\gamma\cdot
\mathbf{u}\right|^2
\end{eqnarray}

\bibliographystyle{acm}
\bibliography{MC}

\end{document}

